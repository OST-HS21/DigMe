\section{Beispiele}

\subsection{Package erstellen}

\begin{lstlisting}
library ieee;
use ieee.std_logic_1164.all;

package parity_package is
	constant nibble : integer; 
	constant word	: integer; -- initialization optional
	function PAR_GEN(AVECT : std_ulogic_vector) return std_ulogic;
end package parity_package;

package body parity_package is
	function PAR_GEN(AVECT : std_ulogic_vector) return std_ulogic is
		variable PO_VAR : std_ulogic;
	begin
		PO_VAR := '1';
		for I in AVECT'range loop
			if AVECT(I) = '1' then
				PO_VAR := not PO_VAR;
			end if;
		end loop;
		return PO_VAR;
	end function PAR_GEN;
	
	constant nibble : integer := 4;
	constant word	: integer := 8;
end package body parity_package;
\end{lstlisting}
\vspace{-20px}


\subsection{Package nutzen}
\begin{lstlisting}
library ieee;
-- declaration of used packages
use ieee.std_logic_1164.all;
use work.parity_package.all;
	
entity parity_gen_8 is
	port( A : in std_ulogic_vector (nibble - 1 downto 0)
		B : in std_ulogic_vector (word - 1 downto 0);
		POA, POB : out std_ulogic);
end entity parity_gen_8;

architecture RTL of parity_gen_8 is
-- no function definition anymore.
-- Function is defined in package.
begin
	POA <= PAR_GEN(A); -- package can be used
	POB <= PAR_GEN(B); -- without further declaration
end architecture RTL;
\end{lstlisting}


