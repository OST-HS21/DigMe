\section{VHDL}
\subsection{Funktionen}
Funktionen haben mehrere Eingänge und einen Ausgang. Es können dadruch zB Operatoren überschrieben werden.
\begin{itemize}
\item Funktionen können rekursiv aufgerufen werden.
\item Funktionen verwenden keine Signalzuweisungen, sondern nur Variablenzuweisung
\item Der Funktionskörper darf keine Warteanweisung oder eine Signalzuweisung enthalten.
\end{itemize}

\begin{lstlisting}
function PARGEN(AVECT : std_ulogic_vector) return std_ulogic is
	variable PO_VAR : std_ulogic;
begin
	PO_VAR := '1';
	return PO_VAR;
end function PARGEN;
\end{lstlisting}

\subsection{Prozeduren}
Prozeduren haben mehrere Ein und Ausgänge. Sie werden häufig für Test-Banches benutzt und sollten nur Vorsichtig für Hardware code verwendet werden.

Beispiel mit 3 Eingängen und 2 Ausgängen:
\begin{lstlisting}
procedure Comp_3(In1,R :in real; Step :in integer; W1,W2 :out real) is
	variable counter: Integer;
begin
	W1 := 1.43 * In1;
	W2 := 1.0;
	L1: for counter in 1 to Step loop
		W2 := W2 * W1;
		exit L1 when W2 > R;
	end loop L1;
	assert (W2 < R)report "Out of range" severity Error;
end procedure Comp_3;
\end{lstlisting}
